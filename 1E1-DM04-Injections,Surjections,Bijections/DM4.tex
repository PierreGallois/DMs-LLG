% to change the appearance of the header, questions, problems or subproblems, see the homework.cls file or
% override the \Problem, \Subproblem, \question or \printtitle commands.

% The hidequestions option hides the questions. Remove it to print the questions in the text.

% CHANGER CE PATH S'IL EST DIFFERENT POUR VOUS
\documentclass[]{../templates/homework}
\usepackage[french]{babel}
\usepackage[T1]{fontenc}
\homeworksetup{
	username={Thomas Diot, Jim Garnier, Jules Charlier, Pierre Gallois \\ 1E1},
	course={Mathématiques},
	setnumber=4}
\begin{document}
\newcommand{\E}{\mathcal E}
\newcommand{\F}{\mathcal F}
\newcommand{\G}{\mathcal G}

\partie {A} {Définitions}
\subproblem

$f: \E \rightarrow \F$ est injective sur $\E$ si et seulement si :
$$\forall x,y \in \E, f(x) = f(y) \implies x = y$$

On en conclut que $f$ n'est pas injective si et seulement si :
$$\exists x,y \in \E, x \neq y \land f(x) = f(y)$$

\subproblem
$f: \E \rightarrow \F$ est surjective de $\E$ sur $\F$ si et seulement si :
$$\forall y \in \F, \exists x \in \E, y = f(x)$$

On en déduit que $f$ n'est pas surjective de $\E$ sur $\F$ si et seulement si :
$$\exists y \in \F, \forall x\in\E, y \neq f(x)$$

\subproblem
$f: \E \rightarrow \F$ est bijective de $\E$ sur $\F$ si et seulement si :
$$\forall y \in \F, \exists ! x \in \E, y = f(x)$$

On en déduit que $f$ n'est pas bijective de $\E$ sur $\F$ si et seulement si :
$$\exists y \in \F,\Bigl(\forall x\in\E, y \neq f(x)\Bigr) \lor \Bigl(\exists a,b\in\E, a\neq b \land f(a) = f(b)\Bigr)$$

\partie {B} {Exemples}
\setcounter{subproblem}{0}
\subproblem

\newpage
\vspace{5cm}

\subproblem

\vspace{12cm}

\subproblem
\question Prouvons que pour tout $n\in\N^{*}$, il existe un unique couple $(p,q) \in \N^{2}$ tel que $f(p,q)= n$.

Si $n=2^{p}(2q+1)$, c'est que $\frac {n} {2^{p}}$ est un entier impair et donc que $2^{p}$ est la plus grande puissance de $2$ qui divise $n$. Par le théorème fondamental de l'arithmétique, ce choix existe, et est unique ($p = v_{2}(n)$). Le choix de $q$ est maintenant forcé par celui de $p$ en prenant l'unique $q\in\N$ tel que $2q+1 = \frac n {2^{p}}$; celui-ci existe, car $\frac n {2^{p}}$ est impair.

\question Par la question précédente, pour tout $n\in\N^{*}$, il existe un unique antécédant de $n$ par $f$. Donc $f$ est bijective de $\N^{2}$ sur $\N^{*}$.

\question Posons $g: \N^{2} \rightarrow \N$ telle que $g(p,q) = f(p,q)-1$. $g$ est injective comme composition de fonctions injectives (voir partie C, question a)), et surjective. En effet, si $n\in\N$, alors $n+1 \in \N^{*}$ a un antécédant par $f$, et $n$ a un antécédant par $g$.

Donc $g$ est une bijection de $\N^{2}$ sur $\N$.

\question i)
\underline{$h$ est injective :}
Soient $(a,b,c), (x,y,z) \in \N^{3}$ deux triplets d'entiers. Si $h(a,b,c) = h(x,y,z)$, par injectivité de $g$, $g(a,b) = g(x,y)$ et $c = z$. Encore par injectivité de $g$, on trouve que $(a,b,c) = (x,y,z)$. Donc $h$ est injective.

\underline{$h$ est surjective :}
Soit $n\in\N$. Alors $n$ a un antécédant $(m,c)$ par $g$ par surjectivité de $g$. Encore par surjectivité de $g$, $m$ a un antédédant $(a,b)$ par $g$. On a donc trouvé des entiers $a,b,c$ tels que $g(g(a,b),c) = n$. Donc $n$ a un antécédant par $h$ ($(a,b,c) \in \N^3$), et $h$ est surjective.

Donc $h$ est une bijection de $\N^{3}$ sur $\N$.

ii) L'antécédant de $2023$ par $g$ est $(3, 126)$. L'antécédant de $2023$ par $h$ est donc $(2,0,126)$.

\question
On prouve le résultat généralisé suivant :
\begin{thm}
  Pour tout $n\in\N^{*}$, il existe une bijection $\varphi_{n}: \N^{n}\xrightarrow{\sim} \N$ ($\N^{n} \cong \N$).
\end{thm}
\begin{proof}
  La preuve est par récurrence. Pour $n\in\N^{*}$, notons $P(n)$ s'il existe une telle bijection $\varphi_{n}$.

  \underline{Initialisation :} Si $n=1$, $\varphi_{1} = \mathrm{Id}_{\N}$ est une bijection de $\N$ sur $\N$. Donc $P(1)$.

  \underline{Hérédité :} Supposons qu'il existe une bijection $\varphi_{n}: \N^{n}\xrightarrow{\sim} \N$ pour $n\in\N^{*}$. Posons $$\varphi_{n+1}(a_{1},\dots,a_{n+1}) = g\left(\varphi_{n}(a_{1}, \dots, a_{n}), a_{n+1}\right)$$

  Prouvons que $\varphi_{n+1}$ est bijective.

  \textbf{Injectivité :} Soient $(a_{1},\dots, a_{n+1}), (b_{1},\dots, b_{n+1})\in\N^{n+1}$. Si $\varphi_{n+1}(a_{1},\dots,a_{n+1}) = \varphi_{n+1}(b_{1},\dots,b_{n+1})$, par injectivité de $g$, puis de $\varphi_{n}$, $(a_{1},\dots, a_{n+1}) = (b_{1},\dots, b_{n+1})$. Donc $\varphi_{n+1}$ est injective.

  \textbf{Surjectivité :} Soit $n\in\N$. Alors $n$ a un antécédant $(m, a_{n+1})$ par $g$. De même, $m \in \N$ et $m$ a un antécédant $(a_{1}, \dots, a_{n})$ par $\varphi_{n}$, car celle-ci est surjective dans $\N$. Donc $n$ a pour antécédant $(a_{1}, \dots, a_{n+1})$ par $\varphi_{n+1}$. Donc $\varphi_{n+1}$ est surjective dans $\N$.

  Donc $P(n+1)$. Par le principe de récurrence, $P(n)$ est vraie pour tout $n\in \N^{*}$.
\end{proof}
En particulier, on obtient que $\varphi: \N^{4} \rightarrow \N$ définie par $\varphi(a_{1},a_{2},a_{3},a_{4}) = g(h(a_{1},a_{2},a_{3}),a_{4})$ est une bijection.

\subproblem
On procède par analyse-synthèse.

\underline{Analyse :} Soit $f$ une fonction qui satisait l'énoncé. Prouvons par récurrence sur $n\in\N$ que $f(n) = n$.

\textbf{Initialisation :} $f(0) \leq 0$, et $f(0) \geq 0$ car $f(0) \in \N$. Donc $f(0) = 0$.

\textbf{Hérédité :} Supposons qu'il existe $n\in\N$ tel que $f(k) = k$ pour tout $k \leq n$. Alors $f(n+1) > n$ : en effet, si $f(n+1) \leq n$, alors $f(n+1) = f(k)$ pour $k = f(n+1) \leq n$, ce qui contredit l'injectivité de $f$.

Donc $f(n+1) \geq n+1$ et $f(n+1) \leq n+1$. Donc $f(n+1) = n+1$.

Par le principe de récurrence, $f(n) = n$ pour tout $n\in\N$.

\underline{Synthèse :} La fonction $f:n\mapsto n$ satisfait bien l'énoncé car elle est injective et $f(n) = n \leq n$. La fonction $f:n \mapsto n$ est donc la seule solution injective de cet énoncé.


\partie {C} {Propriétés}
\setcounter{subproblem}{0}
\subproblem
\question

\underline{$f$ et $g$ injectives :} Soient $x,y\in\E$. Si $g\circ f(x) = g\circ f(y)$, alors par injectivité de $g$, $f(x) = f(y)$. Encore par injectivité de $f$, $x = y$. Donc $g\circ f$ est injective si $f$ et $g$ sont injectives.

\underline{$f$ et $g$ surjectives :} Soit $x \in \G$. Alors $x$ a un antécédant $y\in\F$ par $g$ car $g$ est surjective. De même, $y$ a un antécédant $z\in\E$ par $f$. Donc $x$ a un antécédant, $z$, par la fontion $g\circ f$ : celle-ci est donc surjective si $f$ et $g$ sont surjectives.

\question

\underline{$g\circ f$ injective :} On procède par contraposée en montrant que si $f$ n'est pas injective, alors $g\circ f$ ne l'est pas non plus.

\vspace{2cm}

\underline{$g\circ f$ surjective :} On procède encore par contraposée en supposant que $g$ n'est pas surjective. Alors il existe $x\in\G$ qui n'a pas d'antécédant par $g$. Alors $x$ ne peut pas avoir d'antécédant par $g\circ f$. Donc $g\circ f$ n'est pas surjective.

\subproblem

\question
\underline{1e implication :} Supposons que $g$ est injective. Soient $f_{1},f_{2}:\E\rightarrow\F$ telles que $g\circ f_{1} = g\circ f_{2}$. Si $x\in\E$ par injectivité de $g$, $f_{1}(x) = f_{2}(x)$ pour tout $x\in\E$. Ainsi, si $g\circ f_{1} = g\circ f_{2}$, alors $f_{1} = f_{2}$.

\underline{2e implication :} On procède par contraposée en supposant que $g$ n'est pas injective.

$G$ ne peut être vide : sinon, $g$ est immédiatement injective. Donc $F$ n'est pas vide non plus, et contient au moins deux éléments, sinon quoi $g$ est injective.

Donc il existe $x\in\G$ qui a deux antécédants distincts $a,b\in\F$ par $g$ ($\F$ contient au moins deux éléments car une fonction définie sur un ensemble d'un élément ne peut qu'être injective). Posons $f_{1}(e) = a$ et $f_{2}(e) = b$ pour tout $e\in\E$. Alors $g\circ f_{1} = g\circ f_{2}$, mais $f_{1}(e) \neq f_{2}(e)$ pour tout $e\in\E$ - donc pour au moins un $e\in\E$ car $\E$ n'est pas vide.

\question
\underline{1e implication :} Supposons que $f$ est surjective. Soient $g_{1},g_{2}: \F\rightarrow \G$ telles que $g_{1}\circ f = g_{2}\circ f$. Soit $x \in \F$. Alors il existe $y\in\E$ tel que $f(y) = x$. Comme $g_{1}\circ f(y) = g_{2}\circ f(y)$, on a $g_{1}(x) = g_{2}(x)$ pour tout $x\in\F$. Donc $g_{1} = g_{2}$.

\underline{2e implication :} On procède de nouveau par contraposée en supposant que $f$ n'est pas surjective. Alors il existe $x_{0}\in\F$ qui n'a pas d'antécédant par $f$. Posons
\begin{eqnarray}
  g_{1}(x) =
  \begin{cases}
    a \text{ si } x\neq x_{0} \\
    b \text{ si } x = x_{0}
  \end{cases}
  &\text{  et  }& g_{2}(x) = a
\end{eqnarray}
Où $a,b \in\G$ sont distincts (existent par hypothèse). Alors $g_{1}$ et $g_{2}$ diffèrent en $x=x_{0}$, mais $g_{1}\circ f = g_{2}\circ f$.

On en déduit que si $f$ n'est pas surjective, alors l'assertion $g_1 \circ f = g_2 \circ f \hspace{1cm} g_1 = g_2$ est fausse.

\newpage

\subproblem
Soient $x,y \in \E$ tels que $f(x) = f(y)$. Si $x>y$, alors $f(x) > f(y)$ si $f$ est strictement croissante, $f(x) < f(y)$ si $f$ est décroissante, c'est une contradiction. Similairement, si $y>x$, $f(y) > f(x)$ ou $f(y) < f(x)$ qui est une contradiction. Donc $x=y$ et $f$ est injective.

\end{document}
