% to change the appearance of the header, questions, problems or subproblems, see the homework.cls file or
% override the \Problem, \Subproblem, \question or \printtitle commands.

% The hidequestions option hides the questions. Remove it to print the questions in the text.

% CHANGER CE PATH S'IL EST DIFFERENT POUR VOUS
\documentclass[]{../templates/homework}
\usepackage[french]{babel}
\usepackage[T1]{fontenc}

% Définitions de fonctions latex :

\newcommand{\poly}[4]{\ensuremath{#1x^3 + #2x^2 + #3x + #4}}
\newcommand{\dppoly}[2]{\ensuremath{#1^3 + 3#1^2#2 + 3#1#2^2 + #2^3}}
\newcommand{\dmpoly}[2]{\ensuremath{#1^3 - 3#1^2#2 + 3#1#2^2 - #2^3}}

\newcommand{\dmpolytwo}[2]{\ensuremath{#1^2 - 2#1#2 + #2^2}}

\homeworksetup{
	username={Thomas Diot, Jim Garnier, Jules Charlier, Pierre Gallois \\ 1E1},
	course={Mathématiques},
	setnumber=5}
\begin{document}

\partie {A} {Méthode de Cardan}

\begin{equation}
	\tag{$E_0$}
	\poly{}{a}{b}{c} = 0
	\label{replace_x_by_X}
\end{equation}

\subproblem

\begin{equation*}
	X = x + \frac{a}{3} \quad \Leftrightarrow \quad x = X - \frac{a}{3}
\end{equation*}
On remplace $x$ dans \eqref{replace_x_by_X}.
\begin{align*}
	 &                 & \left(X-\frac{a}{3}\right)^3 + a\left(X-\frac{a}{3}\right)^2 + b\left(X-\frac{a}{3}\right) + c                               & = 0                                      \\
	 & \Leftrightarrow & \dmpoly{X}{\left(\frac{a}{3}\right)} + a\left(\dmpolytwo{X}{\left(\frac{a}{3}\right)}\right) + bX - \frac{ab}{3} + c         & = 0                                      \\
	 & \Leftrightarrow & X^3 - aX^2 + aX^2 + bX +\frac{a^2}{3}X - 2\left(\frac{a^2}{3}X\right) - \frac{a^3}{3^3} + \frac{a^3}{3^2} - \frac{ab}{3} + c & = 0                                      \\
	 & \Leftrightarrow & X^3 + \left(b - \frac{a^2}{3}\right)X + \frac{2a^3}{3^3} - \frac{ab}{3} + c                                                  & = 0                                      \\
	 & \Leftrightarrow & X^3 + pX + q                                                                                                                 & = 0 \tag{$E_1$} \label{equation_with_pq}
\end{align*}
avec $p, q \in \mathbb{R}$ tel que
$\left\{
	\begin{array}{ll}
		p = b - \frac{a^2}{3} \\
		q = \frac{2a^3}{3^3} - \frac{ab}{3} + c
	\end{array}
	\right.$

\subproblem
\question

\begin{equation*}
	X = u + v \quad \text{avec} \quad u, v \in \mathbb{R}
\end{equation*}

On remplace $X$ dans \eqref{equation_with_pq}.
\begin{align*}
	 &                 & (u + v)^3 + p(u + v) + q                & = 0                                      \\
	 & \Leftrightarrow & \dppoly{u}{v} + pu + pv + q             & = 0                                      \\
	 & \Leftrightarrow & u^3 + v^3 + v(3uv + p) + u(3uv + p) + q & = 0                                      \\
	 & \Leftrightarrow & u^3 + v^3 + (u + v)(3uv + p) + q        & = 0 \tag{$E_2$} \label{equation_with_uv}
\end{align*}

\question

On impose $3uv + p = 0$ donc \eqref{equation_with_uv} devient :

\begin{equation*}
	u^3 + v^3 + q = 0 \tag{$E_3$} \label{equation_with_uv_required}
\end{equation*}

\question

\begin{align*}
	 &                 & 3uv + p & = 0                & \text{(relation imposée)}                                           \\
	 & \Leftrightarrow & uv      & = \frac{-p}{3}     &                                                                     \\
	 & \Leftrightarrow & u^3v^3  & = \frac{-p^3}{3^3} & \text{(on élève au cube)} \tag{$E_4$} \label{equation_with_uv_cube}
\end{align*}

\question

\begin{align*}
	 &                 & u^3 + v^3 + q                 & = 0 & \text{d'après \eqref{equation_with_uv_required}} \\
	 & \Leftrightarrow & u^3 + \frac{u^3v^3}{u^3} + q  & = 0 & \tag{$\ast$} \label{multiply_u_cube_to_fraction} \\
	 & \Leftrightarrow & u^3 + \frac{-p^3}{3^3u^3} + q & = 0 & \text{d'après \eqref{equation_with_uv_cube}}     \\
	 & \Leftrightarrow & u^6 - \frac{p^3}{3^3} + qu^3  & = 0 & \text{(on multiplie par $u^3$)}
\end{align*}

Avec $U = u^3$ on a :
\begin{equation}
	\tag{$E_5$}
	U^2 + qU - \left(\frac{p}{3}\right)^3 = 0
	\label{equation_second_degree_with_U}
\end{equation}

qui est une équation du second degré, $u^3$ est donc solution d'une équation du second degré.

À l'étape \eqref{multiply_u_cube_to_fraction} nous aurions pu multipler et diviser par $v^3$ sur $u^3$, nous aurions alors obtenu une équation similaire à \eqref{equation_second_degree_with_U} mais avec $v$ à la place de $u$.

$u^3$ et $v^3$ sont donc solution d'une fonction du second degré.
\end{document}
