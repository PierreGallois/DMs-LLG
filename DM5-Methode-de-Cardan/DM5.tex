% to change the appearance of the header, questions, problems or subproblems, see the homework.cls file or
% override the \Problem, \Subproblem, \question or \printtitle commands.

% The hidequestions option hides the questions. Remove it to print the questions in the text.

% CHANGER CE PATH S'IL EST DIFFERENT POUR VOUS
\documentclass[]{../templates/homework}
\usepackage[french]{babel}
\usepackage[T1]{fontenc}

% Définitions de fonctions latex :

\newcommand{\poly}[4]{\ensuremath{#1x^3 + #2x^2 + #3x + #4}}
\newcommand{\dppoly}[2]{\ensuremath{#1^3 + 3#1^2#2 + 3#1#2^2 + #2^3}}
\newcommand{\dmpoly}[2]{\ensuremath{#1^3 - 3#1^2#2 + 3#1#2^2 - #2^3}}

\newcommand{\dmpolytwo}[2]{\ensuremath{#1^2 - 2#1#2 + #2^2}}

\homeworksetup{
	username={Thomas Diot, Jim Garnier, Jules Charlier, Pierre Gallois \\ 1E1},
	course={Mathématiques},
	setnumber=5}
\begin{document}

\partie {A} {Méthode de Cardan}

\begin{equation}
	\tag{$E_0$}
	\poly{}{a}{b}{c} = 0
	\label{replace_x_by_X}
\end{equation}

\subproblem

\begin{equation*}
	X = x + \frac{a}{3} \quad \Leftrightarrow \quad x = X - \frac{a}{3}
\end{equation*}
On remplace $x$ dans \eqref{replace_x_by_X}.
\begin{align*}
	 &                 & \left(X-\frac{a}{3}\right)^3 + a\left(X-\frac{a}{3}\right)^2 + b\left(X-\frac{a}{3}\right) + c                               & = 0                                      \\
	 & \Leftrightarrow & \dmpoly{X}{\left(\frac{a}{3}\right)} + a\left(\dmpolytwo{X}{\left(\frac{a}{3}\right)}\right) + bX - \frac{ab}{3} + c         & = 0                                      \\
	 & \Leftrightarrow & X^3 - aX^2 + aX^2 + bX +\frac{a^2}{3}X - 2\left(\frac{a^2}{3}X\right) - \frac{a^3}{3^3} + \frac{a^3}{3^2} - \frac{ab}{3} + c & = 0                                      \\
	 & \Leftrightarrow & X^3 + \left(b - \frac{a^2}{3}\right)X + \frac{2a^3}{3^3} - \frac{ab}{3} + c                                                  & = 0                                      \\
	 & \Leftrightarrow & X^3 + pX + q                                                                                                                 & = 0 \tag{$E_1$} \label{equation_with_pq}
\end{align*}
avec $p, q \in \mathbb{R}$ tel que
$\left\{
	\begin{array}{ll}
		p = b - \frac{a^2}{3} \\
		q = \frac{2a^3}{3^3} - \frac{ab}{3} + c
	\end{array}
	\right.$

\subproblem
\question\begin{equation*}
	X = u + v \quad \text{avec} \quad u, v \in \mathbb{R}
\end{equation*}

On remplace $X$ dans \eqref{equation_with_pq}.
\begin{align*}
	 &                 & (u + v)^3 + p(u + v) + q                & = 0                                      \\
	 & \Leftrightarrow & \dppoly{u}{v} + pu + pv + q             & = 0                                      \\
	 & \Leftrightarrow & u^3 + v^3 + v(3uv + p) + u(3uv + p) + q & = 0                                      \\
	 & \Leftrightarrow & u^3 + v^3 + (u + v)(3uv + p) + q        & = 0 \tag{$E_2$} \label{equation_with_uv}
\end{align*}

\question On impose $3uv + p = 0$ donc \eqref{equation_with_uv} devient :

\begin{equation*}
	u^3 + v^3 + q = 0 \tag{$E_3$} \label{equation_with_uv_required}
\end{equation*}

\question

\begin{align*}
	 &                 & 3uv + p & = 0                & \text{(relation imposée)}                                           \\
	 & \Leftrightarrow & uv      & = \frac{-p}{3}     &                                                                     \\
	 & \Leftrightarrow & u^3v^3  & = \frac{-p^3}{3^3} & \text{(on élève au cube)} \tag{$E_4$} \label{equation_with_uv_cube}
\end{align*}

\question

\begin{align*}
	 &                 & u^3 + v^3 + q                 & = 0 & \text{d'après \eqref{equation_with_uv_required}} \\
	 & \Leftrightarrow & u^3 + \frac{u^3v^3}{u^3} + q  & = 0 & \tag{$\ast$} \label{multiply_u_cube_to_fraction} \\
	 & \Leftrightarrow & u^3 + \frac{-p^3}{3^3u^3} + q & = 0 & \text{d'après \eqref{equation_with_uv_cube}}     \\
	 & \Leftrightarrow & u^6 - \frac{p^3}{3^3} + qu^3  & = 0 & \text{(on multiplie par $u^3$)}
\end{align*}

Avec $U = u^3$ on a :
\begin{equation}
	\tag{$E_5$}
	U^2 + qU - \left(\frac{p}{3}\right)^3 = 0
	\label{equation_second_degree_with_U}
\end{equation}

qui est une équation du second degré, $u^3$ est donc solution d'une équation du second degré.

À l'étape \eqref{multiply_u_cube_to_fraction} nous aurions pu multiplier et diviser par $v^3$ sur $u^3$, nous aurions alors obtenu une équation similaire à \eqref{equation_second_degree_with_U} mais avec $v$ à la place de $u$.

$u^3$ et $v^3$ sont donc racines d'un polynôme du second degré.

\partie{B}{Exemples}
\subproblem*{1}
Nous avons $x^3 + 6x + 2 =0$. Remarquons qu'il n'y a pas de terme $x^2$, et posons donc $x = u+v$, avec  $u,v \in \mathbb R$. On obtient :
$$x^3 + px + q = 0, \text{avec }  p=6, q=2$$
Ainsi, en posant $3uv + 6 = 0$, on a :
\begin{align*}
	      uv &= -\frac{p}{3} = -2 \\
	\iff u^3v^3 &= -\frac{p^3}{3^3}= -8
	u^3 + v^3 = -q = -2
\end{align*}
D'où, en posant $U = u^3$, on obtient le trinôme du second degré suivant : 
\begin{align*}
	&U^2 + qU - \left(\frac{p^3}{3^3}\right)=0 \\
	et &\Delta = \sqrt{6^2}
\end{align*}

D'où $U=-4$ ou $U=2$, donc $u=\sqrt[3]{-4}$ ou $u=\sqrt[3]{2}$.

On obtient finalement une unique valeur de $x$ : $x=\sqrt[3]{2}-\sqrt[3]{4}$.

L'équation $x^3 + 6x + 2=0$ n'a donc qu'une seule solution réelle. 
\subproblem*{2}
\begin{align*}
	& 2x^3 + 5x^2 - 24x - 63 = 0 \\
	\iff & x^3 + \frac{5}{2}x^2 - 12x - \frac{63}{2} = 0
\end{align*}

Posons $X = x + \frac{5}{6} \iff x = X - \frac{5}{6}$.

On obtient : $$X^3 - \frac{169}{12}X - \frac{4394}{216} = 0$$

Posons ensuite $X = u + v$, avec $u, v \in \R$ et $3uv - \frac{169}{12} = 0$. On obtient :
\begin{align*}
	&\begin{cases}
		uv = \frac{169}{36}\\
		u^3 + v^3 = \frac{4394}{216}
	\end{cases} \\
	\iff 
	&\begin{cases}
		u^3v^3 = (\frac{169}{36})^3\\
		u^3 + v^3 = \frac{4394}{216}
	\end{cases}
\end{align*}

D'où, avec $U = u^3$, on a :
\begin{align*}
	&U^2 - \frac{4394}{216}U + \left( \frac{169}{36} \right)^3 = 0 \\
	\Delta = 0 \text{ donc } &U = \frac{4394}{216} \times \frac{1}{2} = \frac{2197}{216} \\
	\text{d'où } &u^3 = v^3 = \frac{2197}{216}
\end{align*}
\begin{align*}
	\text{donc } X &= u + v \\
	&= 2\sqrt[3]{\frac{2197}{216}} \\
	&= \frac{13}{3} \\
	\text{d'où } x_1 &= \frac{13}{3} - \frac{5}{6} \\
	&= \frac{7}{2}
\end{align*}
Nous obtenons une première solution. On peut ainsi factoriser le polynôme par $(x - \frac{7}{2})$. Ainsi :
\begin{align*}
	(E') & \iff \left( x - \frac{7}{2} \right) (2x^2 + 12x + 18) = 0 \\
	& \iff 2 \left( x - \frac{7}{2} \right) (x+3)^2 = 0 \\
	& \iff x - \frac{7}{2} = 0 \text{ ou } (x+3)^2 = 0 \\
	& \iff x = \frac{7}{2} \text{ ou } x = -3
\end{align*}
Les solutions de $(E')$ sont donc bien au nombre de deux.
$$ S = \left\{ -3 ; \frac{7}{2} \right\} $$

\subproblem*{3}
$$ x^3 + 3x - 4 = 0$$
\question Une solution évidente est 1 : $$1^3 + 3\times1 - 4 = 0$$
\question Nous n'avons pas de terme en $x^2$, donc posons directement $x = u+v$, avec $u, v \in \R$.
On obtient : $$x^3 + px + q = 0 \text{ avec } p=3, q=-4$$
En posant $3uv + 3 = 0$, on obtient : 
\begin{align*}
 uv = -\frac{p}{3} \iff u^3v^3 = -1 \\
et u^3 + v^3 = -q = 4
\end{align*}
D'où avec $U = u^3$, on obtient : $$U^2 -4U -1 =0$$
On trouve ainsi $U = 2 - \sqrt{5}$ ou $U = 2 + \sqrt{5}$
Or $u^3 + v^3 = -q$, on trouve finalement $x=1$ ou $x = 2 + \sqrt{5} + \sqrt[3]{2 - \sqrt{5}}$ ou $x = 2 - \sqrt{5} + \sqrt[3]{2 + \sqrt{5}}$
\partie C {Formules}
\subproblem*{1} Notons $P$ le polynôme de l'équation $(E_1)$. On sait qu'il existe une racine $X_0 \in \R$ de $P$ si et seulement si il existe $u,v$ tels que $u^3,v^3$ sont les racines du polynôme $D(x) = x^2 + qx - \left(\frac p 3\right)^3$.
Ce polynôme a pour discriminant :
\begin{align*}
	\Delta &= q^2 + 4\left(\frac p 3\right)^3 \\
	&=  \frac {27q^2 + 4p^3}{27}
\end{align*}


Ainsi, si $\Delta > 0$, le polynôme a deux racines distinctes $U,V = \frac {-q}{2} \pm \frac {\sqrt \Delta} {2}$, qui donnent le choix unique de :
$$u,v = \sqrt[3]{\frac {-q}{2} \pm \frac {\sqrt \Delta} {2}}$$

Et enfin le choix unique de :
$$X_0 = \sqrt[3]{-\frac {q}{2} + \sqrt {\frac {27q^2 + 4p^3}{108}} } + \sqrt[3]{-\frac {q}{2} - \sqrt {\frac {27q^2 + 4p^3}{108}} }$$

Qui est donc la seule racine réelle de $P$. \footnote{Nous avons conscience que cet argument n'est pas très rigoureux, mais la méthode de l'exercice suivant n'a pas porté ses fruits sur celui-ci pour prouver l'unicité de $X_1$...}

\subproblem Si $\Delta = 0$, $D$ a une racine double $U = -\frac q {2}$. On doit donc poser $u = v = \sqrt[3]{-\frac q 2}$. Donc $X_1 = 2u$ est une racine de $(E_1)$.


En faisant la division euclidienne de $P$ par $x-X_1$, on trouve $P(x) = (x-X_1)Q(x)$ avec un deuxième facteur quadratique $Q(x) = x^2 + X_1x + (X_1^2 + p)$ dont le discriminant $\Delta'$ est :
$$\Delta' = -3X_0^2 - 4p$$

On trouve enfin que :
\begin{align*}
	\Delta' = 0 &\iff -12u^2 = 4p \\
	&\iff -u^2 = \frac p 3\\
	&\iff -(\sqrt[3]{-\frac q 2})^2 = \frac p3\\
	&\iff -\frac {q^2} 4 = (\frac p 3)^3\\
	&\iff \Delta = 0
\end{align*}
Comme $\Delta = 0$, $\Delta' = 0$ et $Q$ a une racine double $X_2 = -\frac {X_1} 2$.
Ainsi, si $\Delta = 0$, les seules racines de $P$ sont :
\begin{equation*}
	X_1 = 2\sqrt[3]{-\frac q 2} \qquad X_2 = \sqrt[3]{\frac q 2}
\end{equation*}
\end{document}
