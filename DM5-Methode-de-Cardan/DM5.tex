% to change the appearance of the header, questions, problems or subproblems, see the homework.cls file or
% override the \Problem, \Subproblem, \question or \printtitle commands.

% The hidequestions option hides the questions. Remove it to print the questions in the text.

% CHANGER CE PATH S'IL EST DIFFERENT POUR VOUS
\documentclass[]{../templates/homework}
\usepackage[french]{babel}
\usepackage[T1]{fontenc}

% Définitions de fonctions latex :

\newcommand{\poly}[4]{\ensuremath{#1x^3 + #2x^2 + #3x + #4}}
\newcommand{\dmpoly}[2]{\ensuremath{#1^3 - 3#1^2#2 + 3#1#2^2 - #2^3}}

\newcommand{\dmpolytwo}[2]{\ensuremath{#1^2 - 2#1#2 + #2^2}}

\homeworksetup{
	username={Thomas Diot, Jim Garnier, Jules Charlier, Pierre Gallois \\ 1E1},
	course={Mathématiques},
	setnumber=5}
\begin{document}
	
\partie {A} {Méthode de Cardan}

\begin{equation}
	\tag{$E_0$}
	\poly{}{a}{b}{c} = 0
	\label{replace_x_by_X}
\end{equation}

\subproblem

\begin{equation*}
	X = x + \frac{a}{3} \quad \Leftrightarrow \quad x = X - \frac{a}{3}
\end{equation*}
On remplace $x$ dans \eqref{replace_x_by_X}.
\begin{align*}
	&& (X-\frac{a}{3})^3 + a(X-\frac{a}{3})^2 + b(X-\frac{a}{3}) + c &= 0 \\
	&\Leftrightarrow &\dmpoly{X}{(\frac{a}{3})} + a(\dmpolytwo{X}{(\frac{a}{3})}) + bX - \frac{ab}{3} + c &= 0 \\
	&\Leftrightarrow &X^3 - aX^2 + aX^2 + bX +\frac{a^2}{3}X - 2(\frac{a^2}{3}X) - \frac{a^3}{3^3} + \frac{a^3}{3^2} - \frac{ab}{3} + c &= 0 \\
	&\Leftrightarrow &X^3 + (b - \frac{a^2}{3})X + \frac{2a^3}{3^3} - \frac{ab}{3} + c &= 0 \\
	&\Leftrightarrow &X^3 + pX + q &= 0
\end{align*}
avec $p, q \in \mathbb{R}$ tel que
$\left\{
	\begin{array}{ll}
		p = (b - \frac{a^2}{3})\\
		q = \frac{2a^3}{3^3} - \frac{ab}{3} + c
	\end{array}
\right.$

\subproblem


\end{document}